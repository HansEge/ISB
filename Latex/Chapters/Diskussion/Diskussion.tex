%!TEX root = ../../Main.tex
\graphicspath{{Chapters/Diskussion/}}
%-------------------------------------------------------------------------------


\section{Diskussion}
Igennem processen i projektet, mødte gruppen flere problemstillinger. Noget af det første var, at bestemme hvilket filter der skulle bruges for at løse opgaven. Efter søgen på nettet og gennem undervisningsbogen, blev et LMS filter valgt som den rette løsning. Hertil er der blevet produceret et filter, som er testet både i Matlab som simulering, og på blackfin processoren som realisering. 

Igennem testen oplevede gruppen flere forskellige problemstillinger som vil redegøres for herunder. 

\begin{description}[align=left]
\item [Første sample = 1.] Igennem testen blev den første sample sat til et i cross-core koden, dette gøres for at få det samme signal fra input til output. Dette betyder at fejlen allerede på første sample er stor, mens hvis første sample er 0, vil filteret lige så stille og roligt få en større fejl, og derved have en indjusteringstid. Derfor giver det bedre mening at bruge en værdi på en. Dog ser vi en at fejlen e(n) bliver dårligere hvis den første værdi i matlab koden er en og ikke 0. Vi har ikke nogen god forklaring på hvorfor dette er tilfældet, vi mener dog at første værdi burde være en for at sende det samme signal igennem. 
\item [Filter koefficienter] har stor betydning for filteret, især når vi kigger på matlab koden. For at filtrere ordentlig på de forskellige toner og food processeren, skal filteret have 256 filter koefficienter. Modsat høres væsentlig forskel på blackfin, når vi komemr over 32 koefficienter, hvor den omtalte "klik" lyd forværes hvis vi kommer på 64 eller derover. Gruppen mener at dette skyldes at programstrukturen ændrer disse koefficienter på en gang, derfor giver det en kraftigere "klik" lyd når vi overstiger et bestemt antal. Vi har dog ikke nogen forklaring på hvorfor matlab modellen ikke kan filtrere støjsignalerne fra ved lav filter koefficienter. Det burde være muligt at filtrere helt ned til lave filterkoefficienter, især hvis man tester med 3 rene sinus toner.   
\item [Simulering af realiseringen.] 
Da der blev simuleret det eksakte som vi realiserede på blackfin, var resultaterne meget mærkelig ref figur\ref{fig:Tjek_af_frek}, da det ikke ligner at funktionen e(n) = d(n) - y(n) bliver gennemført ordentlig. Hvis man kigger på skaleringen burde filteret trække støj tonerne fra det oprindelige signal. Den bedste forklaring vi havde på dette var at der bliver brugt for lidt koefficienter, så filteret ikke lavede et filter på den rigtige frekvens. Dette tjekkede vi dog i matlab, og fandt at de var ens, derfor kunnes den teori afvises. 
\item [Food processor realisering] Da vi skulle teste om vores filter kunne filtrere på et mere komplekst signal end tre toner, har vi valgt at bruge en foodprocessor. Støjen fra foodprocessoren indeholder mange forskellige frekvenser, hvilket gør kravene til filteret højere. Da vi er begrænset til kun at have 32 koefficienter begrænser det hvor mange af disse frekvenser der kan dæmpes. Dette er en af grundene til at filteret ikke har haft den ønsket effekt. 
En anden grund kan også være at vores timing ikke har været god nok mellem støj signalet og tale-signalet. Da vi testede med enkelte toner fandt vi ud af at bare en lille ændring i støjsignalets frekvens gjorde filteret betydeligt dårligere til at dæmpe støjen. Så hvis timingen ikke har været præcis nok, har de forskellige frekvenser for støjsignalet ikke passet med støjen på talesignalet.
\item ["klik" lyde.] Som tidligere beskrevet opleves der "klik" lyde når filteret køres på blackfin. Den eneste gode forklaring vi har på dette, er at filterkoefficienterne ændres så hurtigt at de 'ødelægger' lyder med "klik" lyde. 
\item [Bedst udenfor 300-3400 Hz.] Vi fandt også ud af gennem projektet at LMS filteret fungerer bedst hvis det støjende signal (x(n)) og det ønskede signal (e(n)) ligger i forskellige frekvenser. Dette ses af figur \ref{fig:Filter_food}, hvor vi har et tale signal, som normalt ligger imellem 300-3400 Hz, sammensat af et food processor som støjer på et meget bredt spektrum. Det ses at filteret skaber et gain af støjen som er hensigten, når vi kommer over 3400 Hz. Dette betyder også at filteret støjsignaler fra som ligger indenfor talesignalet. Dog har vi gennem projektet set at en ren sinus tone på en bestemt godt kan filtreres fra selvom den ligger inderfor tale båndet. Dette skyldes højst sandsanligt at tonen er så kraftig at filteret negligere talesignalet. 
\end{description}